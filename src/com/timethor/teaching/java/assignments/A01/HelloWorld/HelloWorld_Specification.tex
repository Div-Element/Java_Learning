\documentclass{report}

\usepackage{fancyhdr} % Required for custom headers
\usepackage{lastpage} % Required to determine the last page for the footer
\usepackage{extramarks} % Required for headers and footers
\usepackage[usenames,dvipsnames]{color} % Required for custom colors
\usepackage{graphicx} % Required to insert images
\usepackage{listings} % Required for insertion of code
\usepackage{lipsum}
% Margins
\topmargin=-0.45in
\evensidemargin=0in
\oddsidemargin=0in
\textwidth=6.5in
\textheight=9.0in
\headsep=0.25in

\linespread{1.1} % Line spacing

% Set up the header and footer
\pagestyle{fancy}
\lhead{\hmwkAuthorName} % Top left header
\chead{\hmwkDesDiff$\varepsilon$\hmwkDesTime\ : \hmwkTitle} % Top center head
\rhead{\firstxmark} % Top right header
\lfoot{\lastxmark} % Bottom left footer
\cfoot{} % Bottom center footer
\rfoot{Page\ \thepage\ of\ \protect\pageref{LastPage}} % Bottom right footer
\renewcommand\headrulewidth{0.4pt} % Size of the header rule
\renewcommand\footrulewidth{0.4pt} % Size of the footer rule

\setlength\parindent{0pt} % Removes all indentation from paragraphs

%----------------------------------------------------------------------------------------
%	CODE INCLUSION CONFIGURATION
%----------------------------------------------------------------------------------------

\definecolor{MyDarkGreen}{rgb}{0.0,0.4,0.0} % This is the color used for comments
\lstloadlanguages{Perl} % Load Perl syntax for listings, for a list of other languages supported see: ftp://ftp.tex.ac.uk/tex-archive/macros/latex/contrib/listings/listings.pdf
\lstset{language=Java, % Use Perl in this example
        frame=single, % Single frame around code
        basicstyle=\small\ttfamily, % Use small true type font
        keywordstyle=[1]\color{Blue}\bf, % Perl functions bold and blue
        keywordstyle=[2]\color{Purple}, % Perl function arguments purple
        keywordstyle=[3]\color{Blue}\underbar, % Custom functions underlined and blue
        identifierstyle=, % Nothing special about identifiers                                         
        commentstyle=\usefont{T1}{pcr}{m}{sl}\color{MyDarkGreen}\small, % Comments small dark green courier font
        stringstyle=\color{Purple}, % Strings are purple
        showstringspaces=false, % Don't put marks in string spaces
        tabsize=5, % 5 spaces per tab
        %
        % Put standard Perl functions not included in the default language here
        morekeywords={rand},
        %
        % Put Perl function parameters here
        morekeywords=[2]{on, off, interp},
        %
        % Put user defined functions here
        morekeywords=[3]{test},
       	%
        morecomment=[l][\color{Blue}]{...}, % Line continuation (...) like blue comment
        numbers=left, % Line numbers on left
        firstnumber=1, % Line numbers start with line 1
        numberstyle=\tiny\color{Blue}, % Line numbers are blue and small
        stepnumber=5 % Line numbers go in steps of 5
}

% Creates a new command to include a perl script, the first parameter is the filename of the script (without .pl), the second parameter is the caption
\newcommand{\Java}[2]{
\begin{itemize}
\item[]\lstinputlisting[caption=#2,label=#1]{#1.java}
\end{itemize}
}

%----------------------------------------------------------------------------------------
%	DOCUMENT STRUCTURE COMMANDS
%	Skip this unless you know what you're doing
%----------------------------------------------------------------------------------------

% Header and footer for when a page split occurs within a problem environment
\newcommand{\enterProblemHeader}[1]{
\nobreak\extramarks{#1}{#1 continued on next page\ldots}\nobreak
\nobreak\extramarks{#1 (continued)}{#1 continued on next page\ldots}\nobreak
}

% Header and footer for when a page split occurs between problem environments
\newcommand{\exitProblemHeader}[1]{
\nobreak\extramarks{#1 (continued)}{#1 continued on next page\ldots}\nobreak
\nobreak\extramarks{#1}{}\nobreak
}

\setcounter{secnumdepth}{0} % Removes default section numbers
\newcounter{homeworkProblemCounter} % Creates a counter to keep track of the number of problems

\newcommand{\homeworkProblemName}{}
\newenvironment{homeworkProblem}[2][Part \arabic{homeworkProblemCounter}]{ % Makes a new environment called homeworkProblem which takes 1 argument (custom name) but the default is "Problem #"
\stepcounter{homeworkProblemCounter} % Increase counter for number of problems
\renewcommand{\homeworkProblemName}{#1} % Assign \homeworkProblemName the name of the problem
\section{\homeworkProblemName : #2} % Make a section in the document with the custom problem count
\enterProblemHeader{\homeworkProblemName} % Header and footer within the environment
}{
\exitProblemHeader{\homeworkProblemName} % Header and footer after the environment
}

\newcommand{\problemAnswer}[1]{ % Defines the problem answer command with the content as the only argument
\noindent\framebox[\columnwidth][c]{\begin{minipage}{0.98\columnwidth}#1\end{minipage}} % Makes the box around the problem answer and puts the content inside
}

\newcommand{\homeworkSectionName}{}
\newenvironment{homeworkSection}[1]{ % New environment for sections within homework problems, takes 1 argument - the name of the section
\renewcommand{\homeworkSectionName}{#1} % Assign \homeworkSectionName to the name of the section from the environment argument
\subsection{\homeworkSectionName} % Make a subsection with the custom name of the subsection
\enterProblemHeader{\homeworkProblemName\ [\homeworkSectionName]} % Header and footer within the environment
}{
\enterProblemHeader{\homeworkProblemName} % Header and footer after the environment
}

%----------------------------------------------------------------------------------------
%	NAME AND CLASS SECTION
%----------------------------------------------------------------------------------------

\newcommand{\hmwkDesDiff}{A} % Difficulty
\newcommand{\hmwkDesTime}{01} % Time Factor
\newcommand{\hmwkTitle}{Fun with Hello\ World!} % Assignment title

\newcommand{\hmwkCreationDate}{Monday,\ July\ 19,\ 2013} % Due date
\newcommand{\hmwkUpdateDate}{\today} % Due date

\newcommand{\hmwkAuthorName}{Nicholas `Timethor' Rich} % Your name

%----------------------------------------------------------------------------------------
%	TITLE PAGE
%----------------------------------------------------------------------------------------

\title{
\vspace{2in}
\textmd{\textbf{\hmwkDesDiff$\varepsilon$\hmwkDesTime\ :\ \hmwkTitle}}\\
\normalsize\vspace{0.1in}\small{Created\ on\ \hmwkCreationDate}\\
\normalsize\vspace{0.1in}\small{Updated\ on\ \hmwkUpdateDate}\\
\vspace{3in}
Authored by:
}
\author{\textbf{\hmwkAuthorName}}
\date{}
%----------------------------------------------------------------------------------------

\begin{document}

\maketitle

%----------------------------------------------------------------------------------------
%	TABLE OF CONTENTS
%----------------------------------------------------------------------------------------

%\setcounter{tocdepth}{1} % Uncomment this line if you don't want subsections listed in the ToC

\newpage
\tableofcontents
\newpage

%----------------------------------------------------------------------------------------
%	INTRO
%----------------------------------------------------------------------------------------

\section{Overview}
\hspace{2em}\large{This assignment will be an exercise in getting used to Java output. You will be writing one of the most popular introductory programs in the world. Hello World!}\\

\section{Background}
\hspace{2em}\large{There are many methods in Java that will help your program talk to a user. There are three main versions that output to the console and one that creates a small window with an Ok button so you can close it.\\

\textbf{Console Output}\hrulefill\\

\texttt{System.out.print("Hello there");}

\hspace{2em}This one outputs what is in between the parenthesis without going to a new line after.\\

\texttt{System.out.println("Hello there");}

\hspace{2em}This one outputs what is in between the parenthesis and then goes to the next line.\\

\texttt{System.out.printf("\%s", "Hello there");}

\hspace{2em}This one allows you to format your output. For each \%s, \texttt{printf()} expects a new parameter. For example:
\texttt{System.out.printf("\%s \%s", "Hello", "there");}\\
prints the same as the one above. Notice there is a space between the two \%s's this time. Essentially, the \%s's are replaced with each of the following inputs. So, the first \%s is replaced with "Hello" and the second is replaced with "there". Interesting right?\\

\textbf{Windowed Output}\hrulefill\\

\texttt{JOptionPane.showMessageDialog(null, "Hello there")}
 
\hspace{2em}This one creates a small window that displays the message given. Running this multiple times will open multiple windows.\\
}

\normalsize{}

%----------------------------------------------------------------------------------------
%	PART 1
%----------------------------------------------------------------------------------------

\begin{homeworkProblem}{Basic Hello World!}

\homeworkSection{Program}
First this first part, you will be using \texttt{System.out.print();} for your output.

\homeworkSection{Requirements}
You need to use a different \texttt{print()} for each word and punctuation mark. In total that is three \texttt{print()}s.

\homeworkSection{Notice!}
See how using \texttt{print()} puts everything on a single line? Next we will explore how to put text on a new line.

\homeworkSection{Example Output}
\begin{lstlisting}
Hello World!
\end{lstlisting}
\end{homeworkProblem}

%----------------------------------------------------------------------------------------
%	PART 2
%----------------------------------------------------------------------------------------

\begin{homeworkProblem}{Hello World! on separate lines}

\homeworkSection{Program}
Now, add to what you already have. Use \texttt{System.out.println();} for new your output.

\homeworkSection{Requirements}
You need to use a different \texttt{println()} for each word and punctuation mark. In total that is three \texttt{println()}s to add. 

Also, You should have a line from the previous part that looks like: \\
\texttt{System.out.print("!");} \\
change it to: \\
\texttt{System.out.print("!\textbackslash n");} \\

The "\textbackslash n" that you added will turn \texttt{print()} into \texttt{println()}

\homeworkSection{Example Output}
\begin{lstlisting}
Hello World!
Hello
World
!
\end{lstlisting}
\end{homeworkProblem}

\newpage
%----------------------------------------------------------------------------------------
%	PART 3
%----------------------------------------------------------------------------------------

\begin{homeworkProblem}{Hello World! using formatting}

\homeworkSection{Program}
Again, add to what you already have. This time, use \texttt{System.out.printf();} for new your output.

\homeworkSection{Requirements}
Just use one \texttt{printf()} for this part. This time for the format part, use three \%s's and give \texttt{printf()} the extra input "Hello", "World", "!". See the background section if you are confused.

The "\textbackslash n" that you added will turn \texttt{print()} into \texttt{println()}

\homeworkSection{Example Output}
\begin{lstlisting}
Hello World!
Hello
World
!
Hello World!
\end{lstlisting}
\end{homeworkProblem}


%----------------------------------------------------------------------------------------
%	PART 4
%----------------------------------------------------------------------------------------

\begin{homeworkProblem}{Hello World! in a window}

\homeworkSection{Program}
Again, add to what you already have. This time, use \texttt{JOptionPane.showMessageDialog()} for new your output.

if you are not using netbeans or another IDE, you will need to add an import at the top of the file like this: \texttt{import javax.swing.JOptionPane;}

\homeworkSection{Requirements}
This one can be simple. Just use JOptionPane to make a window that says Hello World!.

The "\textbackslash n" that you added will turn \texttt{print()} into \texttt{println()}

\homeworkSection{Example Output}
Console:
\begin{lstlisting}
Hello World!
Hello
World
!
Hello World!
\end{lstlisting}
Window:
\begin{lstlisting}
Hello World!
\end{lstlisting}
\end{homeworkProblem}
\newpage
%----------------------------------------------------------------------------------------
%	BONUS
%----------------------------------------------------------------------------------------

\section{BONUS: More \texttt{printf()}}

\homeworkSection{Program}
Enough of Hello World!, let's have some real fun! Comment out everything you have so far. This part will take a bit of thinking if you have never done anything like this before, but nothing you can't accomplish!

\homeworkSection{Requirements}
Write a short story of about 6 lines or copy paste from something of your choice. Find a way to format it so that it looks good when the console prints it i.e. It is wrapped to a certain number of characters. Try using \texttt{printf} to do this. So you get used to using it as it will be important for later assignments.

\homeworkSection{Hints}
\begin{itemize}
\item The "\textbackslash n" that you added to \texttt{print()} can also be added to \texttt{printf()} in the format string like "\%s\textbackslash n" to get the same effect.
\item You will need to make a String variable to store your story.
\item Strings have the methods \texttt{length()} and \texttt{substring()} that can be helpful for wrapping the story.
\item Strings also have the methods \texttt{startsWith()} and \texttt{endsWith()} that can be helpful for adding a dash ("-") if needed at the end of a line.
\item You want to create a loop that will end when your story has been broken into wrapped lines and printed.
\end{itemize}

\homeworkSection{Example Output}
Console:
\begin{lstlisting}
Lorem ipsum dolor sit amet, consectetuer adipiscing elit. Ut
 purus elit, vestibulum ut, placerat ac, adipiscing vitae, f-
elis. Curabitur dictum gravida mauris. Nam arcu libero, nonu-
mmy eget, consectetuer id, vulputate a, magna. Donec vehicul-
a augue eu neque. Pellentesque habitant morbi tristique sene-
ctus et netus et malesuada fames ac turpis egestas.
\end{lstlisting}
%----------------------------------------------------------------------------------------

\end{document}